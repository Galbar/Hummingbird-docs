\chapter{Context}\label{ch:context}
My interest in game development started almost at the same time as my interest in 
programming.

First, I started experimenting with Game Maker\cite{GameMaker}, where I was able to easily make 
games by basically drag-and-dropping icons that represented changes in the state 
of the game. After a lot of trial and error and messing with its scripting language 
too, I decided to try another tool: Unity Game Engine.
Unity offered a more complex environment but it also allowed for more complex things
. I started learning it by doing tutorials and eventually made a few mini-games.

At that point I wanted to implement games in a much lower level. I wanted to understand 
the insides of how a game works. I started programming games in C++ without using 
any game developing framework or engine (only using SFML\cite{SFML} for multimedia and rendering
). The first time was very complicated, but the more games I developed, the better 
I saw a common pattern in their structure. I always had a main loop\cite{gameloop} with a game 
state update and render; some kind of object pool; some kind of input handling; and 
some kind of rendering pipeline. On the other hand, I was almost never able to reuse 
them because I like trying new technologies, libraries or methods, and most of the 
time these common structures were entangled with calls to the specific libraries 
I was using at the moment.

Eventually, I told to myself, why not implement all these functionalities once, in 
a way that is reusable for my future projects? This is where this project comes into 
place, to implement a flexible and extensible game development framework.
