\chapter{Anatomy of a game}\label{ch:anatomy_of_a_game}

\epigraph{A video game is an electronic game that involves human interaction with 
a user interface to generate visual feedback on a video device such as a TV screen 
or computer monitor.}{Wikipedia}

The quote above is Wikipedia's definition for \textit{video game}. From it we 
can extract the various parts that form a video game:

\begin{itemize}
\item It's a game;
\item Involves human interaction (input);
\item From the input generate some feedback;
\item The feedback is visual (video).
\end{itemize}

In this document we won't tackle the topic of \textit{what makes a game a game}. Instead 
we'll focus on the other three points, which are specific to video games.

The first thing to clear up is that a video game, \textit{game} from now on for simplicity, 
consists of an input-feedback loop. If it wasn't a loop a game would only last one 
frame.

We'll call this input-feedback loop \textbf{main loop} or \textbf{game loop} and 
in its simplest form it looks like the following:

\pagebreak
\begin{lstlisting}[caption=Basic main loop.]
while (true)
{
  processInput(); // Get the human interaction
  update(); // generate the feedback
  render(); // display the feedback
}
\end{lstlisting}

This is an extremely simplified version of a game loop. More complex versions may 
include sleep time between cycles, to lower the power consumption, or include a fixed 
cycle rate, which is useful for physics and AI simulation.

\section{Actors and Scenes}

As games grow more complex, proper structuration of data becomes more important, not 
only for efficiency but mainly for easying the development process. Generally, this 
structure is given by the division of the game in \textbf{scenes} and the representation 
of the game world through \textbf{actors} (or game objects).

Scenes are the natural division of a game. A scene can be a level, a menu screen, 
etc. Usually, they set up the configuration needed for that given part of the game, 
when they start, and, when done, they handle the clean up.

Actors, on the other hand, are the objects that live in the game world. From the player 
to the rock blocking the player's path. They, as their name suggests, \textit{act} 
in the game's world simulation.
