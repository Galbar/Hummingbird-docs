\chapter{State of the art}\label{ch:state_of_the_art}

The functionalities of the game framework I want to develop for this project do not 
conform a full game engine, but they are just a part of what a game engine has to 
offer. For this reason, I considered convenient to study some commercial game engines. 
There exist generic game engines and game engines specialized in specific genres. 
I focused on the generic ones, such as Unity\cite{unity3d} and Unreal Engine
\cite{unrealengine}, because one of my goals is to use it for different projects. 
Both mentioned engines use componentization for giving their actors (game objects 
in Unity) behavior and properties \cite{uecomponents}\cite{unitycomponents}. Unity 
also implements a main loop with a fixed update\cite{unitymainloop}, that is better 
for AI and physics simulations, as it provides stability and allows for simplicity
\cite{gameloop}. On the other hand it is good to render as fast as you can and 
Unity has the rendering outside of the fixed update.

They also implement some kind of resource management that allows easy access to resources 
without having multiple copies of the same. Which is a basic functionality 
for preventing filling the memory with unneeded data.

I also read the on-line book Game Programming Patterns\cite{gameprogrammingpatterns}, 
by Bob Nystrom, to get ideas and solutions to the problems I may face when implementing 
the framework.
