\section{Description of tasks}

In this section I'll list and explain the tasks required to fulfill the project's 
objective. These tasks are sorted by requirements, that is, every task requires the 
one before it to be able to start working on it. For all of the tasks a Laptop PC 
will be needed.

\subsection{Project planning}

This task consist mainly in all the tasks that the GEP course covers. It contains 
the following subtasks:

\begin{itemize}
\item Context and scope definition
\item Temporal planning
\item Budget and sustainability
\end{itemize}

This first task is very important, as it lays out the path for the whole project. 
It makes everything else almost straightforward as the overall view of the project 
is already thought.

The dependencies between the tasks are defined by the order they are presented. That 
means that the task 2 requires the task 1 to be done before being able to work on 
it, and so on with the following tasks.

\subsection{Game framework}

This is the main task for the project. It consist on the development, testing and 
documentation of all the parts that conform the game framework. This main task can 
be divided in the subtasks explained in the following sections.

\subsubsection{Helper classes}

This task consists of developing a set of classes containing general functionality 
for game development. These functionalities consist of describing spatial transformation 
of objects, measuring time and storing intervals of time. These functionalities will 
be encapsulated in the following classes, respectively: \textit{Transformation},
\textit{Clock}, \textit{Time}.

\subsubsection{Actor and Behavior classes}

This task consists of designing the behavior system and implementing the \textit{Behavior} 
virtual class. It also includes implementing the \textit{Actor} class, which will 
mainly be a container for behaviors, and its interface for being able to update the 
game state.


\subsubsection{Game class}

The \textit{Game} class implements the main loop and contains the actor pool. At this stage 
this class won't include the plug-in system, that will be implemented afterwards. 
Moreover, this task will also include extensive testing of all the system up to this 
point, looking for bugs and memory leaks.

\subsubsection{Plug-in system}

After the \textit{Game} class is implemented and thoroughly tested, comes this task. 
It consists in adding the plug-in system to the \textit{Game} class. That is, designing 
and implementing the \textit{Plugin} virtual class and extending the \textit{Game} 
class to handle the plug-ins.

Once this task is done, the framework is completed and ready to use.

\subsection{SFML plug-in (MOGL)}

This task consists in developing a \textit{Plugin} for the game framework that wraps 
the SFML functionality in a way that is useful for developing 2D games with the framework 
and this plug-in.

\subsubsection{Resource Managers}

All games have resources that occupy space in memory such as textures or sounds among 
others. This task consists in developing managers for these resources to handle them 
efficiently.

\subsubsection{Input handler}

User input is what makes a game really a game, otherwise it'd be some kind of movie. 
This task will focus on implementing a data structure that stores the state of the 
input and allows querying that state in a simple way.


\subsubsection{Drawable classes}

This task consists on designing a \textit{Drawable} interface for \textit{Behavior}s 
that have a drawable representation. Then, implement the following drawable behaviors:

\begin{itemize}
\item \textit{Rectangle}, a plain colored rectangle;
\item \textit{Circle}, a plain colored circle;
\item \textit{ConvexShape}, a plain colored convex polygon;
\item \textit{Text}, text using a font;
\item \textit{Sprite}, a static image from a texture;
\item \textit{AnimatedSprite}, a animation from a series of images in a given texture (tile 
sheet).
\end{itemize}

All drawable behaviors will have their own transformation, relative to their actor.

\subsubsection{MultimediaOGL class}

The \textit{MultimediaOGL} class will contain the logic for the rendering pipeline 
of the drawable behaviors, the input handles and the resource managers.

Again, at this point all the previous work will be extensively tested to find possible 
bugs and memory leaks.

\subsection{Final task}

This task is about making sure that all the documentation is done, reviewed to prepare 
the final presentation.
