\section{Methodology and rigor}

This project will be developed using the Agile\cite{agilemanifesto} method with sprints 
of one week. All tasks will be written as tickets in a Trello\cite{trello} board. 
The board will consist of five lists: Backlog, To Do, In Progress, Done, Archive. 
The workflow with this lists is explained bellow.

First, all tasks created are put in the Backlog list. Once a week, every beginning 
of sprint iteration, the progress on the tickets assigned for the ending iteration 
is reviewed and the tickets in Done are moved to Archive.

Next, new tickets from Backlog are moved to To Do, bearing in mind the tickets that 
are still in To Do and In Progress (if any). The tickets put in To Do are the goal 
for the starting sprint and, ideally, will be done before the next one.

During the sprint, when picking a new ticket from To Do, it is moved to In Progress 
to make it clear that this is being worked on. This makes it easier to keep track 
of the progress during the sprint. When a ticket in progress is done, it is moved 
from In Progress to Done.

I chose this system because it allows for fast iteration on the implementation and 
enables me to adapt to changes or unexpected obstacles I may find. It also gives me 
feedback on my progress on-live as it is represented in the Trello board.

The described process will be approached, from a development point of view, using 
Git for version management and the feature branching\cite{featurebranching} technique 
for each ticket. Merging each feature branch into master once the implementation has 
been tested to work properly.  This workflow allows to simultaneously work on multiple 
tickets having each of them encapsulated in their respective branches and ensures 
that master branch never contains broken code.

Finally, to validate that the goals of this project have been achieved it's only needed 
to refer to the scope exposed previously in this document and go through the list 
of features that must be implemented.
