\section{Player}

This class implements the Behavior for the Player (the astronaut). It implements 
the movement, using the keyboard, and displays it on screen. The player has a health 
bar on top of it. The main problem is that we'll be rotating the player actor and thus, 
any Drawable attached to it will also rotate, therefore what we'll do to display the health 
bar is to have another actor that mimics all movement of the player, except for the rotation. 
We'll call this actor the helper.

Below the class code is exposed and explained.

\begin{lstlisting}
#ifndef PLAYER_HPP
#define PLAYER_HPP
#include "hummingbird/hum.hpp"
#include "MOGL/MOGL.hpp"
#include "Bullet.hpp"
#include "Resources.hpp"
#include "math.hpp"

class Player : public hum::Behavior
{
private:
    static int s_lives;
    static float s_vel;
    static unsigned int s_ms_shoot;

    int p_live;
    mogl::MultimediaOGL* p_mogl;
    mogl::AnimatedSprite* p_sprite;
    hum::Kinematic* p_kinematic;
    hum::Kinematic* p_helper_kinematic;
    mogl::Rectangle* p_live_rectangle;
    float p_prev_rotation;
    hum::Clock p_clock;

\end{lstlisting}

We start by including the required files. First, we include Hummingbird and MOGL. 
Then, the various files for the game that are used by the player: Bullet.hpp (the projectile 
shot by the player), Resources.hpp (includes some functions for loading configuration files) and 
math.hpp (includes \texttt{cmath} and implements the method \texttt{square(float x)}, useful for 
calculating distances). We'll see those files in more detail later.

Next, we define the Player class, which is a Behavior, and its private methods.  The 
static variables are used for storing class wise configuration values, read from a 
configuration file on the first instantiation of the class. Then we store useful information, 
such as the number of lives that the player has left and pointers to useful components 
such as the kinematic behavior of the player actor and the helper actor; MOGL, for 
accessing input information; the sprite of the player and the rectangle of the health 
bar; and a clock for measuring time, such as the time between shots.

\begin{lstlisting}
public:
    Player()
    {
        if (s_vel == -1)
        {
            std::stringstream ss;
            readFileContents("res/config/Player.cfg", ss);
            ss
                >> s_vel
                >> s_ms_shoot
                >> s_lives;
        }
    }
\end{lstlisting}

As commented before, the first time a Player is instantiated the configuration is read 
from a file. This avoids having to recomplile the project for adjusting gameplay values.

Afterwards, the initialitazion function of the Behavior is defined. We use it to 
give the player actor all behaviors and configuration needed, such as adding the 
AnimatedSprite to it and creating the helper actor for the health bar.

\begin{lstlisting}
    void init() override
    {
        // get the MOGL plugin instance and store it
        p_mogl = actor().game().getPlugin<mogl::MultimediaOGL>();

        // get the SpriteAnimation, that has already been loaded before
        // game start, and create the AnimatedSprite with it.
        mogl::SpriteAnimation* player_animation =
            p_mogl->spriteAnimations().get("player");
        p_sprite =
            actor().addBehavior<mogl::AnimatedSprite>(player_animation);
        // Set the animation to pause as the player is not moving.
        p_sprite->pause();
        // fix the rotation of the sprite.
        p_sprite->transform().rotation.z = -90;

        // Set the center for transformations of the sprite to the center
        // of the astronaut tile (a bit displaced from the actual center
        // of the 48 by 48 tile)
        p_sprite->setOrigin(hum::Vector3f(24./48., 18./48., 0));

        // Add kinematic behavior to the player actor and store a pointer
        // to it
        p_kinematic = actor().addBehavior<hum::Kinematic>();

        // set other useful information
        p_prev_rotation = 0;
        p_live = s_lives;

        // Create the helper actor we don't need to save a pointer to it,
        // as we will mainly work with its Kinematic Behavior and said
        // behavior has a reference to its actor.
        hum::Actor* helper = actor().game().makeActor();
        // Sync the transformations
        helper->transform() = actor().transform();
        // Add Kinematic, save it and sync it with the player's one.
        p_helper_kinematic = helper->addBehavior<hum::Kinematic>();
        helper->transform().position.z = -0.5;
        helper->transform().position.y -= 0.8;
        helper->transform().position.x -= 0.4;

        // Add the background of the health bar
        auto rect =
            helper->addBehavior<mogl::Rectangle>(sf::Color::White);
        rect->transform().scale = hum::Vector3f(0.8, 0.1, 0.02);

        // Add the foreground color of the health bar (the actual
        // indicator)
        p_live_rectangle =
            helper->addBehavior<mogl::Rectangle>(sf::Color::Green);
        p_live_rectangle->transform().scale =
            hum::Vector3f(0.8, 0.1, 0.02);
        p_live_rectangle->transform().position.z = -0.1;
    }
\end{lstlisting}

Now we need to implement the update method of the player. We will read the input using 
the InputHandler instance inside the MOGL plugin and update the player's speed depending 
on it. We'll also create Bullets if the user presses the left mouse button.

\begin{lstlisting}
    void fixedUpdate() override
    {
        if (isDead())
        {
            p_sprite->stop();
            p_kinematic->velocity().position.x = 0;
            p_kinematic->velocity().position.y = 0;
            p_kinematic->velocity().rotation.z = 0;

            p_helper_kinematic->velocity().position.x =
                p_kinematic->velocity().position.x;
            p_helper_kinematic->velocity().position.y =
                p_kinematic->velocity().position.y;
            return;
        }
\end{lstlisting}

First thing, if the player is dead, do nothing. Otherwise, update the rotation and movement 
speed.

\begin{lstlisting}
        // Look at the mouse
        hum::Vector2f mouse = p_mogl->input().getMouseCurrentPosition();
        mouse /= 48.f;
        float x = mouse.x - actor().transform().position.x;
        float y = mouse.y - actor().transform().position.y;
        float angleInRadians = std::atan2(y, x);
        float angleInDegrees = (angleInRadians / M_PI) * 180.0;
        float delta = angleInDegrees - p_prev_rotation;
        if (delta > 180)
        {
            delta -= 360;
        }
        else if (delta < -180)
        {
            delta += 360;
        }
        p_kinematic->velocity().rotation.z =
            delta / actor().game().fixedUpdateTime().asSeconds();
        p_prev_rotation = angleInDegrees;
\end{lstlisting}

In the fragment above, we first get the current position of the mouse and we transform 
it to game world coordinates by dividing it by 48. This is because in the main funtion 
we set the window to be 1000 by 1000 pixels, but the viewport to display 1 by 1 squares 
as 48 by 48 pixels on screen.  We do this because, by default, all Drawables are 1 
by 1 of size in the game world. To avoid having to set scales to all of them, we scale 
the view.

Then, we calcultate the angle from the player's position to the mouse position and, from it, 
the speed at which the player is rotating. This is where \texttt{p\_prev\_rotation} becomes useful.

\begin{lstlisting}
        // Movement
        if (p_mogl->input().isKeyDown(sf::Keyboard::A))
        {
            p_kinematic->velocity().position.x =
                -s_vel * (not p_mogl->input().isKeyDown(sf::Keyboard::D));
        }
        else if (p_mogl->input().isKeyDown(sf::Keyboard::D))
        {
            p_kinematic->velocity().position.x = s_vel;
        }
        else
        {
            p_kinematic->velocity().position.x = 0;
        }

        if (p_mogl->input().isKeyDown(sf::Keyboard::W))
        {
            p_kinematic->velocity().position.y =
                -s_vel * (not p_mogl->input().isKeyDown(sf::Keyboard::S));
        }
        else if (p_mogl->input().isKeyDown(sf::Keyboard::S))
        {
            p_kinematic->velocity().position.y = s_vel;
        }
        else
        {
            p_kinematic->velocity().position.y = 0;
        }

        if (p_kinematic->velocity().position.x != 0
            or p_kinematic->velocity().position.y != 0)
        {
            p_sprite->play();
        }
        else
        {
            p_sprite->pause();
        }
\end{lstlisting}

For the movement, we just query the keyboard for the status of the direction keys: 
\textbf{A}, \textbf{S}, \textbf{D}, \textbf{W}.  And depending on the input, set the 
speed of the actor. Finally, if the player is moving, play the animation; pause it 
otherwise.

For shooting we check if the mouse left button is pressed and if the minimum time between shots 
has passed. The time is checked by using the clock.

\begin{lstlisting}
        // Shooting
        if (p_mogl->input().isMouseButtonDown(sf::Mouse::Left)
            and p_clock.getTime().asMilliseconds() >= s_ms_shoot)
        {
            auto bullet = actor().game().makeActor();
            float mod = sqrt(square(x) + square(y));
            x /= mod;
            y /= mod;
            bullet->addBehavior<Bullet>(x, y);
            bullet->transform().position.x =
                actor().transform().position.x + 0.6 * x - 0.14 * y;
            bullet->transform().position.y =
                actor().transform().position.y + 0.6 * y + 0.14 * x;
            p_clock.reset();
        }
\end{lstlisting}

If the conditions for shooting are met, we create a new actor, add to it a Bullet Behavior, 
give a direction to it and place it infront of the player. Lastly, we reset the clock.

\begin{lstlisting}
        p_helper_kinematic->velocity().position.x =
            p_kinematic->velocity().position.x;
        p_helper_kinematic->velocity().position.y =
            p_kinematic->velocity().position.y;

        sf::Listener::setPosition(
            actor().transform().position.x,
            actor().transform().position.y,
            10);
    }
\end{lstlisting}

At the end of the update we sync the helper kinematic and set the position of the listener (sounds).

\begin{lstlisting}
    void hit()
    {
        if (not isDead())
        {
            --p_live;
            p_live_rectangle->transform().scale.x =
                (float)p_live/s_lives * 0.8;
        }
    }

    bool isDead() const
    {
        return (p_live <= 0);
    }
};
float Player::s_vel = -1;
unsigned int Player::s_ms_shoot;
int Player::s_lives;
#endif
\end{lstlisting}

Finally, we implement some functions allowing other classes to interact with the player and 
we initialize the static members.
