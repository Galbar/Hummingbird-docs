\section{Evolution of the implementation}

Throughout the development of the project the design of the framework has suffered changes. 
A few of them are explained below.

\subsection{Removal of Scene}
The initial design of the framework included a class for scenes. After a few iterations 
the Scene class felt pointless, having the only functionality of creating the required 
actors at the start, keeping track of them and finally destroying them.

This functionality can be implemented, with better results, with a plugin that does 
exactly the same. With the advantage that said plugin would be tailored to whatever 
needs the actual implementation of the game uses (p.e. scene file format).

And so, the Scene class was droped.

\subsection{Vector2 and Vector3}
Initialy there were not going to be any kind of vector class. The Transformation class 
would include all needed spatial information and that was it.

When the implementation of MOGL started, more and more the need for a way to represent 
spatial information was needed. Mainly to normalize the representation of said information, 
at one point there were sf::Vector3 and glm::vec3 in the public API of the library.

The decision was made to implement Vector2 and Vector3 in a effort to unify the spatial 
information throughout all the Hummingbird framework and MOGL plugin.

\subsection{Draw using OpenGL}
The initial plan was to make use of SFML's Drawable classes to render. At some point 
the decision was made to render the geometry directly using OpenGL so that future implementations 
of Drawables for 3D geometry were easier.

This change implied a complete refactor of the rendering pipeline, requiring the 
management of shaders and the addition of a camera.
