\subsection{ResourceManagers}

\subsubsection{TextureManager}

A hum::ResourceManager for sf::Texture.

Example:
\begin{lstlisting}[caption=TextureManager example]
mogl::TextureManager tm;
tm.load("cat", "path/to/cat.jpg");
sf::Texture* cat = tm.get("cat");

//...

actor.addBehavior<mogl::Sprite>(cat);

//...

tm.free("cat");
\end{lstlisting}

\subsubsection{SpriteAnimationManager}

A hum::ResourceManager for mogl::SpriteAnimations.

This Resource manager is different because it uses SpriteAnimations as the
data to "load" and stores a copy of the given SpriteAnimation.

Example:
\begin{lstlisting}[caption=SpriteAnimationManager example, label=lst:spriteAnimationManager]
mogl::SpriteAnimation jump{
  // texture containing the tilesheet
  game().getPlugin<mogl::MultimediaOGL>()->textures().get("player"),
  0,  // Horizontal offset of the tilesheet
  0,  // Vertical offset of the tilesheet
  0,  // Horizontal margin between tiles in the tilesheet
  0,  // Vertical margin between tiles in the tilesheet
  48, // Width of a tile in the tilesheet
  48, // Height of a tile in the tilesheet
  {5, 6, 7, 8}, // Sequence of ids of the tiles in the tilesheet to play.
  std::vector(4, hum::Time::seconds(0.3f)) // Sequence of hum::Times for each frame in the animation.
};

mogl::SpriteAnimationManager sam;
sam.load("player_jump", jump);

//...

actor.addBehavior<mogl::AnimatedSprite>(sam.get("player_jump"));

//...

sam.free("player_jump");
\end{lstlisting}

\subsubsection{SoundManager}

A hum::ResourceManager for sf::SoundBuffers.

This Resource manager is different because it overwrites the get() method and
has other extra methods.

The SoundManager not just loads sf::SoundBuffers but also allows to play them
through the method play(). This method returns a std::pair containing the
id of the sound resource used (the one that is playing the required
sf::SoundBuffer) and a pointer to the sf::Sound managing the playback of the
sf::SoundBuffer. Note that a \textit{sound resource} is not the same as a sf::SoundBuffer.

The get() methods are also different. In this manager they are accessed by
sound ids (which are returned by play) and they return a pointer of the given
sf::Sound, or nullptr if the sound has been freed.

Example:
\begin{lstlisting}[caption= SoundManager example]
mogl::SoundManager sm;
sm.load("roar", "path/to/roar.ogg");

//...


// fixedUpdate()
if (event)
{
  roar_id = sm.play("roar", 50).first; // start playing the "roar" sound
}
// onDestroy()
if (sm.get(roar_id) != nullptr) // if the sound is still playing
{
  sm.get(roar_id)->stop(); // stop it
}

//...

sm.free("roar");
\end{lstlisting}

As shown in the example, when a sf::SoundBuffer playback is done (Stopped),
then the \textit{sound resource} is cleared and therefore the sound id invalidated
(getting its related sound returns nullptr).

Sound ids start at 1 and always grow, that means that a sound id of 0 will
always return nullptr.

\subsubsection{MusicManager}

A hum::ResourceManager for sf::Music.

Example:
\begin{lstlisting}[caption=MusicManager example]
mogl::MusicManager mm;
mm.load("music1", "path/to/music1.ogg");
sf::Music* music = mm.get("music1");
music->play();
//...
music->stop();
mm.free("music1");
\end{lstlisting}

\subsubsection{ShaderProgramManager}
A hum::ResourceManager for mogl::ShaderProgram.

This Resource manager is different because it uses ShaderProgramDefs to load
the resource (ShaderProgram) instead of the usual `std::string`.

Example:
\begin{lstlisting}[caption=ShaderProgramManager example]
// Create shaders
mogl::Shader vs, fs;

// Load vertex shader from file
vs.loadFromFile(Shader::Type::VERTEX_SHADER, "shader.vert");
if(!vs.isCompiled())
{
  hum::log_e("Vertex shader failed to compile: ", vs.log());
  return 1;
}

// Load fragment shader from file
fs.loadFromFile(Shader::Type::FRAGMENT_SHADER, "shader.frag");
if(!fs.isCompiled())
{
  hum::log_e("Fragment shader failed to compile: ", fs.log());
  return 1;
}

mogl::ShaderProgramDef def{vs, fs, "out_color"};

mogl::ShaderProgramManager spm;
spm.load("plain_shader", def);

mogl::ShaderProgram* sp = spm.get("plain_shader");
sp->use()->setUniform4f("color", 0, 1, 1, 1);

// ...

spm.free("plain_shader");
\end{lstlisting}
