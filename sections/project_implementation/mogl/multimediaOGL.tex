\subsection{MultimediaOGL}
Plugin that handles all the rendering pipeline, input and resource
managment.

The MultimediaOGL plugin is the class that groups all the functionalities
included in MOGL and makes them accessible in the hum::Game.

\subsubsection{Rendering}

The plugin handles the creation of the sf::Window and the OpenGL context.
The window is made accessible through window().

The plugin also handles the camera, that is accessible by calling getCamera().
To set and get the clear color setClearColor() and getClearColor() can be
called.

The rendering of all active and enabled Drawables happens at every postUpdate.
This may seem anti-intuitive because the position of the Actors is only updated
every fixedUpdate (we'd be drawing the same frame multiple times), but
MultimediaOGL does something smart: for all Drawables whose Actors have a
hum::Kinematic behavior, the position of the Drawable will be interpolated using
the kinematic information and the fixedUpdateLag. Also, AnimatedSprites's animation
is also updated when drawn. This way we can have a smooth as posible view of the
game world and a fixedUpdate for the game logic.

Finally, a game space -> draw space transformation can be set by using
setDrawSpaceTransform(), for more details see the mehtod's details.

\subsubsection{Resource Management}

The MultimediaOGL plugin owns an instance of each of the resource managers
included with MOGL. This way one can access any of them at any point inside the
game.

Example:
\begin{lstlisting}[caption=MultimediaOGL example]
class Player : public hum::Behavior
{
private:
  hum::Kinematic* k;
  mogl::AnimatedSprite* spr;
  mogl::SoundId sound_id;
  mogl::MultimediaOGL* mogl;
public:
  void init() override
  {
    sound_id = 0;
    // Get a pointer to the MultimediaOGL instance in the Game
    mogl = actor().game().getPlugin<mogl::MultimediaOGL>();
    k = actor().getBehavior<hum::Kinematic>();
    spr = actor().getBehavior<mogl::AnimatedSprite>();
    spr->setSpriteAnimation(mogl->spriteAnimations().get("player_idle"));
  }

  void fixedUpdate() override
  {
    ...
    if (k->velocity().x != 0 || k->velocity().y != 0)
    {
      // Get the animation "player_walking" from the resource manager for sprite animations
      auto anim = mogl->spriteAnimations().get("player_walking");
      if (anim != spr->spriteAnimation())
          spr->setSpriteAnimation(anim);

      // Play a sound using the sound manager
      if (mogl->sounds().get(sound_id) == nullptr)
      {
        sound_id = mogl->sounds()->play("steps", 75, true);
      }
    }
    else
    {
      auto anim = mogl->spriteAnimations().get("player_idle");
      if (anim != spr->spriteAnimation())
          spr->setSpriteAnimation(anim);
      if (mogl->sounds().get(sound_id) != nullptr)
      {
        mogl->sounds()->get(sound_id)->stop();
      }
    }
  }
}
\end{lstlisting}
