\subsection{Actor}

This class represents an object in the Game. You can create a new Actor by calling
Game::makeActor(). This method will create a new Actor and return it. The Game owns
the Actor and it controls its lifetime.

To destroy an Actor you must call Game::destroy(), not its destructor.
The Actor then, will be marked to be destroyed and after the next update step
it'll be deleted. Just before being deleted, the Actor will call its
Behaviors onDestroy() method.

An Actor, by default, doesn't have any behaviour and \textbf{you should not inherit
from it}.  Instead Actors are composed by Behaviors. These Behaviors are
the ones that must implement the behaviour of the Actor composed by them.

On the other hand, Actors do have a Transformation, accessible through transform()
and a reference to its Game, accessible through game().

A Actor can be \textbf{active} or \textbf{inactive}. If a Actor is \textbf{inactive} it exists,
all its Behaviors also exist and have been instantiated; but it \textbf{won't}
be updated. Same applies to its Behaviors.

Usage example. The Actor will be destroyed after 10 fixed updates:
\begin{lstlisting}[caption=Actor example]
// We define two Behaviors: A and B.
class B : public hum::Behavior
{
public:
    B(int x): value(x)
    {}

    int value;
}

class A : public hum::Behavior
{
public:
    A(int x): current(x)
    {}

    void init() override
    {
        last = actor().getBehavior<B>()->value;
    }

    void fixedUpdate() override
    {
        current++;
        if (current > last)
        {
            actor().game().destroy(actor());
        }
        hum::log("Count: ", current);
    }

private:
    int last, current;
}

int main()
{
    hum::Game g;
    hum::Actor* a = g.makeActor();
    // here we add the A and B to the Actor.
    A* t = a->addBehavior<A>(1);
    a->addBehavior<B>(10);
    g.run();
    return 0;
}
\end{lstlisting}

