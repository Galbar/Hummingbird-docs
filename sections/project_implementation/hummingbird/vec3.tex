\subsection{Vector3}

hum::Vector3 is very similar to hum::Vector2, the only one difference is that it 
has three dimensions (x, y and z) instead of two. It works the same way Vector2 does 
and implements the same operations.

As for Vector2, you generally don't have to care about the templated form (hum::Vector3<T>),
the most common specializations have special typedefs:
\begin{itemize}
\item hum::Vector3<float> is hum::Vector3f
\item hum::Vector3<int> is hum::Vector3i
\end{itemize}

Usage example:
\begin{lstlisting}[caption=Vec3 example]
hum::Vector3f v1(16.5f, 24.f, 13.f);
v1.x = 18.2f;
float y = v1.y;

hum::Vector3f v2 = v1 * 5.f;
hum::Vector3f v3;
v3 = v1 + v2;

bool different = (v2 != v3);
\end{lstlisting}
