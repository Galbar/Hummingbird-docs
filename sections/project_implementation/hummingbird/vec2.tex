\subsection{Vector2}
hum::Vector2 is a simple class that defines a mathematical
vector with two coordinates (x and y). It can be used to
represent anything that has two dimensions: a size, a point,
a velocity, etc.

The template parameter T is the type of the coordinates. It
can be any type that supports arithmetic operations (+, -, /, *)
and comparisons (==, !=), for example int or float.

You generally don't have to care about the templated form (hum::Vector2<T>),
the most common specializations have special \texttt{typedef}s:
\begin{itemize}
\item hum::Vector2<float> is hum::Vector2f
\item hum::Vector2<int> is hum::Vector2i
\end{itemize}

The hum::Vector2 class has a small and simple interface, its x and y members
can be accessed directly (there are no accessors like setX(), getX()) and it
contains no mathematical function like dot product, cross product, length, etc.

Usage example:
\begin{lstlisting}[caption=Vec2 example]
hum::Vector2f v1(16.5f, 24.f);
v1.x = 18.2f;
float y = v1.y;

hum::Vector2f v2 = v1 * 5.f;
hum::Vector2f v3;
v3 = v1 + v2;

bool different = (v2 != v3);
\end{lstlisting}
