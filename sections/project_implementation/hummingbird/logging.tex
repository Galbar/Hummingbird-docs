\subsection{Logging}

The framework includes a set of methods to help the debug process by allowing to print 
messages depending on the environment (release or debug) and to various channels (standard 
or error).

\subsubsection{assert\_msg()}
Check a condition and if it fails, exit the program and print the message.

This method does nothing if NDEBUG is defined.

Usage example:
\begin{lstlisting}[caption= assert\_msg() example]
hum::assert\_msg(player\_x > 64, "Player is outside of the map! x=", player\_x);
\end{lstlisting}


\subsubsection{log()}
Print a message to the standard output.

T can be any type that has the operator \texttt{<<} overloaded.
It can also be any of the following classes:
\begin{itemize}
\item hum::Vector2
\item hum::Vector3
\item hum::Transformation
\item hum::Time
\item hum::Clock
\end{itemize}

Usage example:
\begin{lstlisting}[caption= log() example]
hum::log("Player position: ", actor().transform().position);
\end{lstlisting}

\subsubsection{log\_e()}
Print a message to the error output.

T can be any type that has the operator \texttt{<<} overloaded.
It can also be any of the following classes:
\begin{itemize}
\item hum::Vector2
\item hum::Vector3
\item hum::Transformation
\item hum::Time
\item hum::Clock
\end{itemize}

Usage example:
\begin{lstlisting}[caption= log\_e() example]
hum::log\_e("Player position: ", actor().transform().position);
\end{lstlisting}

\subsubsection{log\_d()}
Print a message to the standard output.

This method does nothing if NDEBUG is defined.

T can be any type that has the operator \texttt{<<} overloaded.
It can also be any of the following classes:
\begin{itemize}
\item hum::Vector2
\item hum::Vector3
\item hum::Transformation
\item hum::Time
\item hum::Clock
\end{itemize}

Usage example:
\begin{lstlisting}[caption=log\_d() example]
hum::log\_d("Player position: ", actor().transform().position);
\end{lstlisting}
